\documentclass[12pt, letterpaper]{article}
\usepackage[margin=3cm]{geometry}
\usepackage{multicol}
\usepackage{amsmath}
\usepackage{tikz}
\usepackage{pgfplots}
\usepackage{float}
\pgfplotsset{width=7.5cm}

\begin{document}

\begin{center}
	{\LARGE{Relatório 2º projeto ASA 2021/2022}}\\[\baselineskip]
\end{center}

\begin{flushleft}
	\textbf{Grupo:} t20\\
	\textbf{Alunos:} Guilherme Pascoal (99079), Pedro Lobo (99115)
\end{flushleft}


\section{Descrição do Problema e da Solução}

O problema apresentado tem como objetivo, dada uma árvore genealógica
representada por um grafo, determinar se este representa uma árvore válida e,
se sim, determinar o ancestral comum mais próximo entre dois nós.\\

De modo a verificar que uma árvore é válida, processa-se o grafo transposto,
verificado que este novo grafo se trata de um grafo acíclico dirigido em que cada
nó não tem mais do que 2 vizinhos diretos.
Esta verificação é feita com uma DFS: se for encontrado um arco para trás,
estamos na presença de um ciclo, o que torna a árvore inválida. De igual modo,
se um nó tiver mais do que 2 vizinhos diretos, que representam mais do que 2
ancestrais diretos no grafo original, este é invalidado.\\

Caso o grafo corresponda a uma árvore válida, atravessa-se o grafo e marca-se os
vértices atingíveis a partir dos dois nós dados. Estes vértices, atingíveis por
ambos os nós, correspondem aos ancestrais comuns. De seguida, representa-se o
grafo formado apenas pelos ancestrais comuns através do grafo original. Os
ancestrais comuns mais próximos correspondem àqueles que, neste último grafo
não têm nenhum descendente, ou seja, têm um \emph{out degree} igual a 0.\\


\section{Análise Teórica}

\begin{itemize}
	\item Leitura dos dados de entrada. $\mathcal{O}(V + E)$
	\item Validação da árvore, com recurso a DFS. $\mathcal{O}(V + E)$
	\item Marcação dos nós atingíveis através do 1º nó, com recurso a BFS. $\mathcal{O}(V + E)$
	\item Marcação dos nós atingíveis através do 2º nó, com recurso a BFS. $\mathcal{O}(V + E)$
	\item Criação do grafo constituído pelos nós comuns, percorrendo todos os vértices e arestas. $\mathcal{O}(V + E)$
	\item Iteração sobre todos os vértices, apresentando apenas os ancestrais comuns mais próximos. $\mathcal{O}(V)$
\end{itemize}
Complexidade global: $\mathcal{O}(V + E)$


\section{Avaliação Experimental dos Resultados}

Foram gerados grafos aleatórios de tamanho 1 a 1000001 em incrementos de 50000.
Por tamanho do grafo, entende-se a soma do número de vértices com o número de
arestas. Abaixo é possível observar o gráfico:

\begin{figure}[H]
	\centering
	\begin{tikzpicture}[scale=1.0]
		\begin{axis}[
			xmin=0, xmax=1000001,
			ymin=0, ymax=0.8,
			xlabel={Tamanho do grafo (V+E)},
			ylabel={Tempo(s)},
			]
			\addplot[blue,mark=square] table{../benchmarks/output.txt};
		\end{axis}
	\end{tikzpicture}
\end{figure}

O gráfico revela que o tempo de execução do algoritmo cresce linearmente com a
soma do número de vértices com o número de arestas do grafo, o que seria de
esperar de acordo com a análise teórica realizada.\\


\section{Bibliografia}

\begin{itemize}
	\item Thomas H. Cormen, Charles E. Leiserson, Ronald L. Rivest, Clifford Stein (2009)
	\emph{Introduction to Algorithms, Third Edition}, The MIT Press.
\end{itemize}

\end{document}
